\section{Methodology and Model}

We analyse $CO_2$ emissions from the electricity grid before and after the exogenous \ac{DST} change, using a \acf{DiD} framework. To compare our results to the literature, we additionally run the same specification, looking at electricity consumption in $kWH$. As shown in Figure \ref{fig:map}, we specify Queensland, which does not use \ac{DST} as our control and the other states in the \ac{NEM} as the treatment. In order to get plausible coefficients from the \ac{DiD} regression, we first verify the common prior assumption, using an event-study graph.
For the this, we use the \ac{DST} time shift in the respective states as the treatment, with emissions and electricity consumption as the outcomes, as shown in Figure \ref{fig:intraday co2}. We include other controls, and entity and time fixed effects for every state and date to isolate the effect of \ac{DST} from possible confounders. 
The controls include average sunlight irradiance, average wind speed, maximum daily temperature, weekend and public holidays.
\ac{DST} coincides with greater solar power generation, which displaces fossil fuel generation and can potentially reduce emissions. To control for this, average daily sunlight irradiance (adjusting for cloud cover and the Earth's tilt) is used as a control. Similarly, average wind speed is used as a control for wind generation.

Maximum daily temperature is used as a control to explain heating and cooling loads.\footnote{Due to thermal inertia, instantaneous temperature would be a less effective control}
Weather data is aggregated at a daily level because intraday data (especially sun intensity) would be a collider.
Temperature data was taken from capital cities, since most thermal load is consumed in capital cities. Wind and sunlight data was taken from region midpoints, as representative values for generation which is typically dispersed across the region.
Controls were added for weekend and public holidays, because they tend to reduce electricity demand.

To correct for a potentially missing common prior trend and improved interpretability we follow \textcite{kellogg_daylight_2008} and additionally implement a \ac{DDD} design, first performed by \textcite{gruber_incidence_1994}, and the associated event study. The DDD design allows us to correct for unobserved factors affecting the control and treatment groups differently. To establish the \ac{DDD}, we normalise by midday emissions and electricity demand, as midday emissions and electricity demand respectively will be the least affected by the time shift caused by DST, the sun is at it's highest point and thus the effect of the DST time shift is minimal compared to the morning and evening periods.

\subsection{Difference-in-differences (DiD) Estimation}
Following \textcites{callaway_difference--differences_2021, goodman-bacon_difference--differences_2021}, equation \ref{eq:DD} shows the \ac{DiD} regression we implement.
\begin{equation}
    \left(\frac{CO_2}{Population}\right)_{r,t}
 = \beta_0 + \beta_1*Treatment_{r} + \beta_2Post_{t} + \beta_3(Treatment \times Post)_{r,t} + \beta_4 Controls_{r,t}
     + \epsilon_{r,t}
     \label{eq:DD}
\end{equation}

\textit{Treatment} is 1 if a region has \ac{DST} at any time (i.e. all studied regions except Queensland). \textit{Post} is 1 during each \ac{DST} period (October to March). Errors are clustered by region, to mitigate the impact of serial correlation. The data is weighted by population. We also run the same \ac{DiD} with $KwH$ per capita as second outcome.
To examine the common prior trend assumption for both emissions and electricity consumption, \ref{eq:ES-DD} shows the event study performed using the Stata package provided by \textcite{clarke_implementing_2021}.
\begin{equation}
    y_{r,t} = \sum_{j=2}^{J} \beta_{-j} \times (Lead_j)_{r,t} + \sum_{k=0}^K \beta_{k} \times (Lag_k)_{r,t} + \mu_r + \lambda_t + X^{'}_{r,t} + \epsilon_{r,t}
    \label{eq:ES-DD}
\end{equation}
where $y_{r,t}$ is $CO_2$ or electricity production per capita for region $r$ and time period $t$ respectively. The $\beta$ coefficients represents our event study terms for the lead and lag effects of DST respectively, estimating the effect of DST based on the time to treatment $t$, which in our case is the number of days into DST, being negative when DST is not active, 0 when the DST transition occurs, and positive when DST is active. $\mu_r$ and $\lambda_t$ are entity and time-fixed effects by region and day respectively. $X_{r,t}$ represent a vector of controls per region and day and their respective coefficients.  

\subsection{Difference-in-difference-in-difference (DDD) Estimation}
The \ac{DiD} framework allows us to identify the average differences between the treatment and control states. However, it does not control for state-specific demand shifts. To do so, we make use of the fact, that electricity demand during the mid-day does not see a shift from DST, compared to morning and evening peaks.\footnote{compare \textcite{kellogg_daylight_2008}} 
Adding this third difference, we run our DDD regressions (equation \ref{eq:DDD}) with the same data, the same additional outcome of $kWh p.c$, again clustering by region and weighting by population.
\begin{align}
    \label{eq:DDD}
    \left(\frac{CO_2}{Population}\right)_{r,t,m} &= \beta_0 + \beta_1Treatment_{r} + \beta_2Post_{t} + \beta_3NotMidday_{m}   \\
    & +\beta_4(Treatment \times Post)_{r,t} + 
    \beta_5(Treatment \times NotMidday)_{r,m} \nonumber \\ 
    & +\beta_6(Post \times NotMidday)_{t,m} + \beta_7 (Treatment \times Post \times NotMidday)_{r,t,m} \nonumber \\ 
    &+ \beta_8 Controls_{r,t}  + \epsilon_{r,t,m}
    \nonumber 
\end{align}
\textit{NotMidday} is 0 for half hours between 12:00 and 14:30 (local time), and 1 for the remainder, with the subscript $m$ specifying whether the observation takes place during the midday period.
To check the prior common trends assumption we make, we re-run the event study design shown in equation \ref{eq:ES-DD}, adjusting our outcome variables by the midday values for $CO_2$ and electricity production respectively. This is equivalent to taking ratios of the outcome variable allow us to create a quasi-equivalent event study design for the DDD regression we perform, as specified by \textcite{olden_triple_2022}. The main coefficient of interest is $\beta_7$ that captures the effect of not being in the period between 12 and 14:30, in a treatment region, while \ac{DST} is active.
